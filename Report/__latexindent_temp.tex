\documentclass{article}

\usepackage[margin=1.5in]{geometry}


\begin{document}

\title{Games Design Report}
\author{Callum Darling}
\date{May, 2020}
\maketitle
\parskip 0.2in

\section{Introduction}

L

\section{Changes from plan}\label{Changes}

In the original designs for the game I had envisioned an Entity Component System driven design.
The core of the design and the final product are still very similar although there have been a number of smaller changes to the structure of components and Entities.

The largest change has been to one of the design patterns. I will elaborate further on this in Section \ref{Patterns}. 
Having read up more on the singleton pattern I decided
that the context I had planned to use it in was not appropriate for its actual intent. While a useful pattern it should be used sparingly. 
I replaced this pattern with the sequencing pattern Game Loop.
This was due to the fact that I was already implementing a game loop and I didn't feel any of the other design patterns fit well with the overall design. 

When implementing the game it became clear that more entities and components were required than were accounted for in the initial plans. 

The designs had 8 components, the final product has 10, these are due to oversights related to the media library being used.
The way that objects are drawn to the screen in SFML is sufficiently different between some kinds of objects that is was prudent to seperate this functionality into different components.
Aside from this the planned components covered all needed eventualities.

The designs had 11 entities, the final product has 14, these similarly were created to make interfacing with SFML easier.
For example the original designs had not accounted for the ability to draw shapes (rectangles, circles, etc.) directly on the screen without using a sprite with a texture.
To account for this a new entity was created.
The distinction between Entity and Enemy also proved to not be useful in this context.
 While the enemies are still built from a different base to the other entities,
 there was no reason to keep the distinction between the types of object as this can be expressed through components within the entities.

Almost all of the character designs and user interfaces have changed from the initial design, as more complete and polished versions have been produced.

\section{Design Patterns}\label{Patterns}

The three patterns that I chose were the architectural pattern Entity Component System, the design pattern Factory and the sequencing pattern Game Loop. 

\subsection{Replacement of Singleton with Game Loop}

As mentioned in section \ref{Changes} I was initially planning on using the singleton design instead of game loop. 
There are two classes the deal with files in the design, ResourceHandler and LevelHandler, 
initially I believed that due to this file access I did not want multiple instances of these classes existing at once, so I thought that it would be appropriate to use a singleton pattern.
Having made this choice I began researching implementations of this pattern to try and find a good way to use it in my code.
While researching I came across many warnings against the overuse of this pattern and examples of its use in areas that it was not really required. 
Looking at this and also situations that it is more appropriate to use this pattern I realised that it may not be appropriate for this project.
ResourceHandler and LevelHandler are implemented in such a way that there is never more than once instance of either of them operating at once and there is no way for the same file to be simultaneously accessed.
This proves that the singleton pattern would have been unnecessary.

\subsection{Entity Component System}

I used a C++ library called EnTT to implement ECS.
The Components can be found in the  src/components.
Each component is a simple file containing a struct containing variables for the data that the component will encompass.

Entities initialised and stored in a registry (the registry for this project can be found in game.h & game.cpp).
The entity class is simply a reference that can be used with the registry to access components that have been emplaced into the entity itself.
The entities are initialised in the EntityFactory (EntityFactory.h & EntityFactory.cpp), e.g. they are created and have components emplaced into them.


I chose to use the Entity Component System (ECS) architectural pattern for several reasons.
In past projects completed using a traditional inheritance based object oriented approach I had run into a trouble with large inheritance trees. 
I knew that I was going to be using entities of some kind in the game whichever approach is took so it seemed reasonable to use the system that was oriented towards this approach.
ECS systems have a reputation for being resource efficient and fast to execute, I thought this would be interesting to look into.

\subsection{Factory Pattern}

balh

\subsection{Game Loop}

blah

\section{Games Concepts}
\section{Security Requirements}
\section{Reflections}
\section{Test Strategy}

\end{document}}

